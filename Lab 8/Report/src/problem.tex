\CWHeader{Лабораторная работа \textnumero 8}

\CWProblem{
На первой строке заданы два числа, $N$ и $p \ge 1$, определяющие набор монет некоторой страны с номиналами $p^0, p^1, \dots, p^{N-1}$. Нужно определить наименьшее количество монет, которое можно использовать для того, чтобы разменять заданную на второй строчке сумму денег $M \leq 2^{32} - 1$ и распечатать для каждого i-го номинала на i-ой строчке количество участвующих в размене монет. Кроме того, нужно обосновать почему жадный выбор неприменим в общем случае (когда номиналы могут быть любыми) и предложить алгоритм, работающий при любых входных данных.
}
\pagebreak
