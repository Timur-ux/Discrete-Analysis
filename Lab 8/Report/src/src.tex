\section{Описание}
Т.к. каждый следующий номинал делится на предыдущий, то для нахождения оптимального решения можно воспользоваться жадным алгоритмом. Для начала смотрим на максимально возможное количество монет максимального номинала, потом следующего по убыванию и т.д.

В общем случае это решение не будет работать. Например, монеты: 1, 3, 4 и сумма 6. Тогда по жадному алгоритму мы получим 4, 1, 1, а оптимальным решением будет 3, 3.

Для нахождения оптимального решения в общем случае можно воспользоваться динамическим программированием. Пусть cost(C) -- минимальное количество монет, которым можно разменять сумму равную C. Номиналы: $c_1, c_2, \dots, c_n$. Тогда справедливы следующие утверждения:
\begin{enumerate}
  \item $cost(c_i) = 1, i \in \{1, 2, \dots, n\}$ 
  \item $cost(C) = \min_{i \in \{1, 2, \dots, n\}}\{cost(C - c_i)\} + 1$
\end{enumerate}

На основании данных соотношений можно построить динамическое решение данной задачи для монет произвольных номиналов.

\pagebreak

\section{Исходный код}

\begin{minted}{c++}
#include <iostream>

using ll = long long ;
using ull = unsigned long long ;
int main() {
  ull N, p, M;

  std::cin >> N >> p >> M;
  
  ull *values = new ull[N];

  for(size_t i = 0, currentPi = 1; i < N; ++i) {
    values[i] = currentPi;
    currentPi *= p;
  }

  ull * counts = new ull[N] {0};

  for(size_t i = N - 1; i > 0; --i) {
    counts[i] = M / values[i];
    M -= counts[i] * values[i];
  }

  counts[0] = M;

  for(size_t i = 0; i < N; ++i)
    std::cout << counts[i] << '\n';

  delete [] values;
  delete [] counts;
  return 0;
}
\end{minted}

\section{Консоль}
\begin{alltt}
raison@raison-Lenovo:~/git/Discrete-Analysis/Lab 8/Report$ ../../build/Lab\ 8/Lab8_main 
3 5
71
1
4
2
\end{alltt}
\pagebreak
