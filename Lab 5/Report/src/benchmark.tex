\section{Тест производительности}
Как видно из тестовых данных, алгоритм Ахо-Корасик изначально показывает кратно лучшие результаты по сравнению с алгоритмами КМП и z-функцией.

При росте количества паттернов все алгоритмы показывают линейный рост, но если для КМП и z-функции это объясняется увеличением количества паттернов, то рост Ахо-Корасик объясняется увеличением суммарной длины всех паттернов. Это доказывает последний тест, где уменьшились длины паттернов, но их количество осталось неизменным -- Ахо-Корасик ускорился, в то время как остальные алгоритмы замедлились.
\begin{verbatim*}
raison@WIN-4SUTO50B1V5:~/Learn/Discrete-Analysis$ ./out/Debug/main < test.txt
Pattern's quantity: 250
Overall pattern's size: 2205
Text size: 9957
Aho-Korasik time: 20
KMP time: 164
z-function time: 103
raison@WIN-4SUTO50B1V5:~/Learn/Discrete-Analysis$ ./out/Debug/main < test.txt 
Pattern's quantity: 500
Overall pattern's size: 4211
Text size: 9961
Aho-Korasik time: 31
KMP time: 278
z-function time: 216
raison@WIN-4SUTO50B1V5:~/Learn/Discrete-Analysis$ ./out/Debug/main < test.txt 
Pattern's quantity: 1000
Overall pattern's size: 8435
Text size: 9949
Aho-Korasik time: 48
KMP time: 562
z-function time: 429
raison@WIN-4SUTO50B1V5:~/Learn/Discrete-Analysis$ ./out/Debug/main < test.txt
Pattern's quantity: 1000
Overall pattern's size: 1954
Text size: 9952
Aho-Korasik time: 35
KMP time: 604
z-function time: 433
\end{verbatim*}

\pagebreak

