\CWHeader{Лабораторная работа \textnumero 3}

\CWProblem{
Исследование качества программ

Для реализации словаря из предыдущей лабораторной работы, необходимо провести исследование скорости выполнения и потребления оперативной памяти. В случае выявления ошибок или явных недочётов, требуется их исправить.

Результатом лабораторной работы является отчёт, состоящий из:
\begin{enumerate}
  \item Дневника выполнения работы, в котором отражено что и когда делалось, какие средства использовались и какие результаты были достигнуты на каждом шаге выполнения лабораторной работы.
\item Выводов о найденных недочётах.
\item Сравнение работы исправленной программы с предыдущей версией.
\item Общих выводов о выполнении лабораторной работы, полученном опыте.
\end{enumerate}
Минимальный набор используемых средств должен содержать утилиту gprof и библиотеку dmalloc, однако их можно заменять на любые другие аналогичные или более развитые утилиты (например, Valgrind или Shark) или добавлять к ним новые (например, gcov)}
\pagebreak
