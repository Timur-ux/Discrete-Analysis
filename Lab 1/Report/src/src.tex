\section{Описание}
Требуется написать реализацию алгоритма сортировки подсчётом.

Как сказано в [1]: \enquote{основная идея сортировки подсчетом заключается в том, чтобы для каждого входного 
элемента $x$ определить количество элементов, которые меньше $x$}.

В качестве небольшого улучшения сдвинем диапазон входных числа к нулю слева, вычев из всех чисел минимальное.
Это позволит использовать отрицательные числа как индекс в массиве счётчика, а также, если левая граница диапазона больше нуля, сжать размер массива счётчика до размера диапазона, а не максимального числа.

\pagebreak

\section{Исходный код}

\begin{lstlisting}[language=C++]
#include <iostream>

using namespace std;

#ifndef COUNTING_SORT_HPP_
#define COUNTING_SORT_HPP_
/*
    Define class CountingSort that response for sorting
    Define struct Item that will contain key-value pairs
*/
#include <vector>

struct Item {
    unsigned int key;
    unsigned long long value;
};

namespace sorts {

class CountingSort {
private:
    unsigned int minElement_;
    unsigned int maxElement_;

    void checkForMinMaxElements(vector<Item> & items) { // find min and max keys
        minElement_ = items[0].key;
        maxElement_ = items[0].key;
        for (auto & item : items) {
            unsigned int currentKey = item.key;
            if (currentKey < minElement_) {
                minElement_ = currentKey;
            }

            if (currentKey > maxElement_) {
                maxElement_ = currentKey;
            }
        }
    }

public:
    CountingSort() = default;

    void sort(vector<Item> & items) { // sorting
        if (items.size() == 0) {
            return;
        }

        checkForMinMaxElements(items);
        vector<Item> tempItems(items); /*
            Create items's copy 
            Vector-argument will be sorted
        */
        vector<unsigned int> elementsCounter(maxElement_ - minElement_ + 1);

        for (auto & item : items) {
            elementsCounter[item.key - minElement_]++;
        }

        for (int i = 1; i < elementsCounter.size(); ++i) {
            elementsCounter[i] += elementsCounter[i - 1];
        }


        for (int i = tempItems.size() - 1; i >= 0; --i) {
            long index = --elementsCounter[tempItems[i].key - minElement_];
            items[index] = tempItems[i];
        }
    }
};

} // !sorts

#endif // !COUNTING_SORT_HPP_


int main() {
    ios::sync_with_stdio(false); // Setting that busts IOput
    cin.tie(0);

    vector<Item> data;

    unsigned int key;
    unsigned long long value;
    while (cin >> key >> value) {
        data.emplace_back(Item{ key, value });
    }

    sorts::CountingSort sorter;
    sorter.sort(data);

    for (auto & item : data) {
        cout << item.key << '\t' << item.value << '\n'; // Using endl will call flush buffer so i use '\n'
    }

    return 0;
}	
\end{lstlisting}

\section{Консоль}
\begin{alltt}
PS C:/Users/User/Desktop/Learning/2Course/Discrete-Analysis> cmake --build ./build --config Debug --target main -j 10 --
PS C:/Users/User/Desktop/Learning/2Course/Discrete-Analysis> cd build
PS C:/Users/User/Desktop/Learning/2Course/Discrete-Analysis/build> ./main.exe
0	13207862122685464576
65535	7670388314707853312
0	4588010303972900864
65535	12992997081104908288
^D
0	13207862122685464576
0	4588010303972900864
65535	7670388314707853312
65535	12992997081104908288
\end{alltt}
\pagebreak

